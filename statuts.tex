\documentclass[11 pt]{article}
\usepackage[french]{babel}
%\usepackage{draftcopy}
\usepackage{url}
\usepackage{hyperref}
%%% eurosym provides \euro
%\usepackage{eurosym}
%%% Use \pounds for GBP
\usepackage[utf8]{inputenc}
%\usepackage[T1]{fontenc}
\usepackage{fullpage}
\usepackage{titlesec}
\usepackage{graphicx}
\usepackage{caption}

\pagestyle{empty}
\parskip=8 pt
\raggedright

\newcommand\fmm[0]{Français pour une Meilleure Mobilité}

\title{\fmm \\[4mm]
\large Assemblée générale constitutive du 23 septembre 2023}
\date{23 septembre 2023}


\begin{document}

\maketitle

\section{Dénomination}

Il est fondé entre les adhérents aux présents statuts une association
régie par la loi du 1\ier{} juillet 1901 et le décret du 16 août 1901,
ayant pour dénomination: \og\fmm\fg.


\section{Objet}

Cette association a pour objet la défense de l’environnement naturel
en promouvant notamment des solutions et bonnes pratiques visant à
améliorer le déplacement au sein de la métropole nantaise, les Pays de
la Loire et le Grand Ouest, afin que les usagers disposent d’un réel
choix lors de chaque déplacement et d’alternatives efficaces à
l’automobile. Elle mène notamment:

\begin{itemize}
\item des actions visant à apporter de la transparence sur nos choix
  et habitudes de déplacement afin d’enrichir un débat souvent
  empreint d’émotivité par des informations factuelles;
\item un travail de fond d’agrégation, recoupement, extension et
  publication de données publiques sur la question du déplacement, et
\item des campagnes de sensibilisation à l’usage des modes de
  transports actifs et doux,
\end{itemize}
ainsi que toute autre action utile à l’accomplissement de son objet.

L’association exerce principalement son activité au sein de
l’agglomération nantaise. Toutefois, dans le cadre de la mise en
œuvre de l’objet susmentionné, elle peut agir à un niveau régional ou
national, le cas échéant.

\section{Siège social}

Le siège social est fixé au 14 bis rue Metzinger, 44100 Nantes.

\section{Durée}

La durée de l’association est illimitée.

\section{Composition, adhésion, cotisations}

L’association se compose de personnes physiques. Sont adhérents les
personnes à jour de leur cotisation.  Le montant de la cotisation est
adopté par les adhérents lors de l’assemblée générale. Il peut être
révisé annuellement.  L’association ne distingue pas entre adhésion et
don, qui sont la même chose dans son contexte.

Afin de poursuivre ses objectifs, l’association vise un nombre
important d’adhérents.  Dans l’intérêt de l’efficacité dans la conduite
de ses assemblées générales (ordinaires et extraordinaires), les membres
admis à voter lors desdites assemblées sont limités à ceux qui s’impliquent
d’une façon bénévole à l’association ou qui signalent au comité
directeur au moins trois mois avant l’AG le souhait de pouvoir voter.
Le comité directeur peut admettre des demandes faites ultérieurement à
cette date si la logistique de la tenue de l’assemblée générale le
permet.  Cette clause a un intérêt purement logistique et ne doit être
utilisée pour priver des membres du droit de vote.

L’exercice de l’association court du 1\ier{} janvier au 31 décembre.

\section{Radiations}

La qualité de membre se perd par :

\begin{itemize}
\item Le non-renouvellement de la cotisation au-delà de 13 mois après la dernière cotisation;
\item La démission;
\item Le décès;
\item La radiation prononcée par le conseil d’administration pour
  non-paiement de la cotisation ou pour motif grave, l’intéressé ayant
  été invité à fournir des explications devant le bureau et/ou par
  écrit.
\end{itemize}


\section{Ressources}

Les ressources de l’association comprennent :

\begin{itemize}
\item Le montant des cotisations et dons;
\item La vente d’objets et de vêtements, en lien avec l’association
  et/ou ses actions;
\item Toutes les ressources autorisées par les lois et règlements en
  vigueur.
\end{itemize}


\section{Assemblée générale ordinaire}

L’assemblée générale ordinaire comprend tous les membres de
l’association. Elle se réunit chaque année, à une date fixée par le
conseil d’administration.

Un membre du bureau, le président si ce rôle est attribué, assisté des
membres du conseil, préside l’assemblée et expose la situation morale
ou l’activité de l’association.

Le membre du bureau qui agit en fonction de trésorier rend compte de
sa gestion et soumet les comptes annuels (bilan, compte de résultat et
annexe) à l’approbation de l’assemblée.

L’assemblée générale fixe le montant des cotisations annuelles.

Les décisions sont prises à la majorité relative des voix des membres
ayant le droit de vote et présents ou représentés. Un membre ne peut
représenter plus de quatre autres membres.

Il est procédé, après épuisement de l’ordre du jour, au renouvellement
des membres sortants du conseil.

Toutes les délibérations sont prises à main levée sauf sur demande
d’au moins deux membres présents.

Les décisions des assemblées générales s’imposent à tous les membres,
y compris absents ou représentés.


\section{Assemblée générale extraordinaire}

Si besoin est, ou sur la demande d’un quart des adhérents, tout membre
du bureau peut convoquer une assemblée générale extraordinaire,
suivant les modalités prévues aux présents statuts.

\section{Conseil d’administration}

L’association est dirigée par un conseil de 15 membres au maximum,
élus pour 2 années, renouvelés en moitié chaque année par l’assemblée
générale. Les membres sont rééligibles.

En cas de vacance, le conseil pourvoit provisoirement au remplacement
de ses membres. Il est procédé à leur remplacement définitif à la
plus prochaine assemblée générale. Les pouvoirs des membres ainsi élus
prennent fin à l’expiration du mandat des membres remplacés.

Le conseil d’administration se réunit au moins une fois tous les six
mois, sur convocation d’au moins un membre du bureau, ou à la demande
du quart de ses membres.

Les décisions sont prises à la majorité des voix; en cas de partage,
la voix de la majorité des membres du bureau est prépondérante; en cas
de partage du bureau, la décision n’est pas prise mais est soumise à
discussion ultérieure.

Le Conseil d’administration peut déléguer des pouvoirs pour une durée
déterminée, à un ou plusieurs de ses membres.

Tout membre du conseil qui, sans excuse, n’aura pas assisté à trois
réunions consécutives sera considéré comme démissionnaire.


\section{Bureau}

Le conseil d’administration élit parmi ses membres, un bureau qui agit
de manière collégiale.  Le bureau peut élire parmi ses membres un président,
trésorier, ou tout autre rôle nécessaire pour l’achèvement de ses fonctions.

\section{Indemnités}

Toutes les fonctions, y compris celles des membres du conseil
d’administration et du bureau, sont exercées à titre gratuit et bénévole. Seuls les
frais occasionnés par l’accomplissement de leur mandat sont remboursés
sur justificatifs. Le rapport financier présenté à l’assemblée
générale ordinaire présente, par bénéficiaire, les remboursements de
frais de mission, de déplacement ou de représentation.

\section{Règlement Intérieur}

Un règlement intérieur peut être établi par le conseil
d’administration, qui le fait alors approuver par l’assemblée
générale.

Ce règlement éventuel est destiné à fixer les divers points non prévus
par les présents statuts, notamment ceux qui ont trait à
l’administration interne de l’association.

\section{Dissolution}

En cas de dissolution prononcée, un ou plusieurs liquidateurs sont
nommés, et l’actif net, s’il y a lieu, est dévolu à un organisme ayant
un but non lucratif ou à une association ayant des buts similaires,
conformément aux décisions de l’assemblée générale extraordinaire qui
statue sur la dissolution. L’actif net ne peut être dévolu à un membre
de l’association, même partiellement, sauf reprise d’un apport.


\vspace{5 mm}
Fait à Nantes, le 23 septembre 2023

\vspace{1 cm}
Jeffrey Abrahamson

\vspace{1 cm}
Hugo Mougard

\end{document}
